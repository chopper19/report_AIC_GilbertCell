\section{Analysis of a Gilbert cell-based multiplier }

\subsection{Gilbert cell overview}
A CMOS-based technology Gilbert cell-based mixer is shown in figure \textbf{AGGIUNGI IMMAGINE}. This circuit exploits a differential topology to implement a double-balanced cell providing:
\begin{itemize}
	\item Reasonable conversion gain (from one to some tens);
	\item Thanks to the double-balanced topology if performs good rejection of input frequency components to the output port along with high linearity;\footnote{The linearity is meant with respect to the conversion gain.}
	\item Good isolation between ports is provided by the high suppression of spurious frequency components;
	\item Thanks to the CMOS architecture it can be integrated.
\end{itemize}
\subsection{Gilbert cell circuit analysis}
To analyse the circuit it is necessary to give some informations about its topology, polarization and driving. Looking a FIGURA DOVE C'E' MIXER COMPLETO it is possible to identify five main blocks that are described in what follows.
\paragraph{Bias net}

This net is made up of the current mirror (transistors M1 and M2) and the voltage reference generation branch (transistor M5, R1, R2, R3, R4, R5). 

Transistor M1 is the strong branch of the current mirror and it acts as a current sink for the differential pair made up by M3 and M4, fixing the polarization current for the whole circuit. M2 sinks instead the current from the voltage reference section of the bias net.
Generally transistors in current mirrors are polarized in saturation region, so that an high output resistance is seen from the stage above: this means that M1 should appear as a good current sink.
Given $V_{G1}=V_{GS1}$ the gate voltage in M1 we have that the transistor remains in saturation if:
\begin{gather}
\label{eq:sat_m1}
V_{GS}\geq V_{th}  \\
V_{DS}>V_{od} = V_{GS}-V_{th} 
\end{gather}
The same holds for M2, that is always in saturation condition since it is diode connected (provided \ref{eq:sat_m1}). In saturation (neglecting channel modulation effects), one has that the drain current only depends on gate voltage and it is given by:
\begin{equation}
\label{eq:Id_quadLaw}
I_D = \mu_{n,eff} \frac{C_{ox}}{2} \frac{W}{L} (V_{GS}-V_{th})^2 = \frac{\beta_{n}}{2}V_{od}^2
\end{equation}
hence M1 remains in saturation till when:
\begin{equation}
V_{DS1}\geq V_{od1}= V_{th}-\sqrt{\frac{2I_0}{\beta_{n}}}
\end{equation}
therefore to have a low value of the overdrive voltage one should have large transistors, i.e. large W.
The output resistance in instead defined by:
\begin{equation}
 r_o = \frac{1}{\lambda I_0} \propto L
\end{equation}
where $\lambda$ takes count of the channel modulation. It appears that we need long transistors to have better current sink's properties.
By looking at FIGURE 2, one has (supposing both M2 and M1 in saturation, and $V_{GS1}=V_{GS2}=V_{DS1}=V_{DS2}$):
\begin{gather}
I_0 = \frac{\beta_{n1}}{2}(V_{GS1}-V_{th})^2(1+\lambda_1V_{DS1}) \notag \\
I_{REF} = \frac{\beta_{n2}}{2}(V_{GS1}-V_{th})^2(1+\lambda_1V_{GS1}) \notag
\end{gather}
hence, supposing to have equal transistors:
\begin{equation}
\frac{I_0}{I_{REF}} = \frac{W_1/L_1}{W_2/L_2} 
\end{equation}
Ideally the current mirroring is only depending on geometrical parameters, this means that a very good matching is required. Even if the two devices are very close to each other within the same chip, it exists the possibility to have a variation in parameters that depend on temperature, i.e. $V_{th}$ and $\mu_{n,eff}$. Supposing to have equal devices with everything constant but:
\begin{gather}
K_{n1}=K_n+\Delta K_n \notag \\
V_{th,1} = V_{th}+\Delta V_{th} \notag
\end{gather}
one eventually finds that:
\begin{equation}
\frac{I_0}{I_{REF}} \simeq 1 + \frac{\Delta K_n}{K_n}+2\frac{\Delta V_{th}}{V_{od}} \notag
\end{equation}
Then, some of the possible mismatch causes are:
\begin{itemize}
	\item $K_n$, that can varies a lot in case of wide circuits;
	\item $V_{od}$ that usually is set low and varies with $V_{th}$, whose small variation can produce large differences in the two currents;
	\item difference between $V_{DS1}$ and $V_{DS2}$.
\end{itemize}
Assuming as maximum variations:
\begin{gather}
\frac{\Delta K_n}{K_n} = \pm 5 \% \notag \\
\frac{\Delta V_{th}}{V_{GS}-V_{th}} = \pm 10 \% \notag
\end{gather}
one has:
\begin{equation}
\frac{I_0}{I_{REF}} = 1 \pm 15 \% \notag
\end{equation}
A cascaded voltage divider is connected as M2's load: 
\begin{itemize}
	\item M5 is diode connected and it is used to generate the right gate bias voltage for the gain stage.
	\item A resistive voltage divider made up of R2 and R4 is used to polarize the mixing stage.
	\item Resistors R1 and R3 are used to connect the bias net to the gain and mixing stages' gates. They also acts as high impedance for the RF and LO signals coming from outside the circuit and prevents them to enter the bias net.
	\item Capacitors C1 and C2 shunt possible non-DC disturbances coming from the Gilbert Cell, improving the bias isolation.
\end{itemize}

Noise issues will not be analysed within this treatise, it should be considered for a more complete design though. As a rule of thumb, narrow gate and low overdrive voltages should used to lower noise contribution.(FONTE) 
Using a resistive voltage reference can be detrimental in case of voltage supply fluctuation, since a resistive voltage divider cannot reject current variations. However using active devices we would produce much more noise injection within the net though.

\paragraph{Gain stage}

The gain stage (shown in figure 3) is the linear part of the mixer, it must handle without distortion and corruptions the power coming from the RF signal giving some amplification. The topology is suited for a low noise amplifying purpose (FONTE): a differential stage with source degeneration. The stage is made up of two transistors (M3 and M4) in common source configuration directly polarized by M1 and two source degeneration resistances $R_S$.

The RF differential signal modulates the current flowing into each of the two transistors, that has to be polarized in saturation region. For both of them it must be ensured an high output voltage range and a quite low overdrive voltage, along with a large drain to source bias voltage (required to maintain the stage in saturation during the swing of the LO stage).

To have an idea on the gain that can be obtained by this stage, one can look at the stage in FIGURE 3. Removing the degeneration resistances one has that the gate-to-gate mesh gives:
\begin{equation}
V_{in,1}-V_{in,2} = V_{GS3}-V_{GS4} \notag
\end{equation} 
from equation \ref{eq:Id_quadLaw}, recalling that $I_{1}+I_{2}=I_0$ and squaring:
\begin{equation}
(V_{in,1}-V_{in,2})^2 = \frac{2}{\beta_n}(I_0-2\sqrt{I_1 I_2}) \notag
\end{equation}
that once inverted yields
\begin{equation}
I_{1}-I_{2} = \sqrt{\beta_n I_0}(V_{in,1}-V_{in,2})\sqrt{1-\frac{\beta_n}{4I_0}(V_{in,1}-V_{in,2})^2} \notag
\end{equation}
writing the previous equation in term of $\Delta I =I_{1}-I_{2} $ and $\Delta V_{in} = V_{in,1}-V_{in,2}$ and computing the value of the slope ot this characteristic, one obtains that the maximum differential voltage gain in equilibrium condition $\Delta V_{in} =0$ is given by:
\begin{equation}
\label{eq_statiGain}
|A_{v}|= \sqrt{\beta_{n}I_0}R_1
\end{equation}
This suggest that it is better to polarize the stage exactly with  $V_{GS1}=V_{GS2}$. The complete expression for equation \ref{eq_statiGain} also gives that the differential transconductance is a strongly non-linear function of the gate bias voltage. 

As we said before there it is the possibility to have a mismatch between the transistors and the loads due to circuit dimensions, temperature distribution and process tolerances, then looking at FIGURE 3:$M3 \neq M4$ and $R1 \neq R2$. It can be demonstrated that the offset output voltage due to differences in the circuit parameters and dimensions is:
\begin{equation}
V_{o,offset}= \Delta V{th}+\frac{V_{GS}-V_{th}}{2}\Bigg(-\frac{\delta R}{2R}\frac{\Delta W /L}{2W/L}\Bigg) \notag
\end{equation}
To reduce this error a common centroid topology should be used in layout.
Besides, it is important to notice that R1 and R2 are both the source output conductance of a common gate stage (mixing stage of the Gilbert cell) and that the actual gain for the RF stage is a current gain. 

The small signal equivalent transconductance of the stage, once we added the source degeneration, (APPENDICE?) is given by:
\begin{equation}
\label{eq_degenGain}
G_{eq} = \frac{g_m}{1+g_m R_S}
\end{equation}
By adding $R_S$ we lower the RF stage's current gain, however  we also get a more linear behaviour reducing its dependency with respect to bias point (this fact will be demonstrated later). LINEARITA' CON OVERDRIVE?? (RAZAVI rf PAG 193)

\paragraph{Mixing stage}
This stage is also called \emph{switching stage}, suggesting the way we want to drive transistors M6 to M9 (FIGURE 4). Also this one has a differential topology, however, differently from the RF stage, we drive each pair with complementary LO signals that turn on and off cycle-by-cycle one of the two devices. The driving signal il sinusoidal (similarly one can apply a square wave) and it must be large enough to ensure the abrupt switching of the stage: when a device is on it must be in saturation (not in  triode!), while the other one must be completely of, interdicted.
A compromise is needed since:
\begin{itemize}
	\item small amplitude of LO signals produce slower switching speed, hence part of the power is wasted as common mode signal at the output;
	\item too large LO signals drive the MOSFET in triode region producing spikes that corrupt the output signal with unwanted feed-through  and reduces speed.
\end{itemize}
Switching speed is related to the device cut-off frequency, that for short channel devices is inversely proportional to channel length and linearly dependent to the overdrive voltage:
\begin{equation}
f_T \propto  V_{od}/L \notag
\end{equation}
suggesting that the highest overdrive and the minimum channel length should be chosen. Again, a compromise must be found among the following solutions:
\begin{itemize}
	\item Setting the available length ( the minimum is fixed by the used design kit);
	\item Selecting higher overdrive voltage one has the possibility to increase f\textsubscript{T}, however to keep our switches in saturation and to obtain a large output swing we need to keep small this quantity;
	\item Widening the gate width one has faster device, but the source to bulk capacitance increases, eventually shunting the RF signal coming from the LNA stage;
	\item Reducing the current faster commutation is achieved, along with lower transconductance though.
\end{itemize}

Even if the switching stage is not linear it is possible to define its gain and in order to do that we analyse one single pair. We recall that the gate LO signal modulates the stage drain current as a square wave, that can be written in term of its Fourier's series:
\begin{equation}
 i_D(t)=I_{pk}\Bigg(\frac{1}{2}-\frac{2}{\pi} \sum_{n=1,3,5 \dots}^{} \frac{1}{n}\sin(n\omega t)\Bigg) \notag
\end{equation} 
where only odd harmonics are present. The instantaneous gain fo the stage, related to the first harmonic is:
\begin{equation}
\label{eq_SwitchGain}
A_c|_{switch} = \frac{2}{\pi}
\end{equation}
that is neither dependent on the bias of the stage nor on the actual device transconductance.\footnote{It is however important to remember what stated above, regarding the intensity of the applied gate voltage, that further reduces this quantity in case of too large input signals. Moreover its better to notice that the stage introduces a loss rather than an actual gain.} It is important to notice that equation \ref{eq_SwitchGain} is exact in case of gate's square wave signals, whereas in case of sinusoidal driving the actual gain will be reduced because of the smoother variation of the drain current waveform.

\paragraph{Load stage} 
The load stage is formed by two resistors R\textsubscript{L} whose role is to give the stage enough gain and to set the output bias voltage through the RF drain current. They must be as matched as possible to make the differential outputs of the Gilbert cell equals.


\subsection{Mathematical analysis: conversion gain}

As already known a mixer is a non-linear time variant system, hence a descriptive approach based on the behaviour of the cell is well suited to find the conversion gain of the cell.\footnote{It is not possible to use neither the superposition of effect nor the Laplace transform to compute the gain due to the presence of the switching pairs.} 
The following analysis is based on the schematic in \textbf{FIGURE 6}.

We suppose all the transistors in gain stage and bias in saturation, then we identify:
\begin{itemize}
	\item i\textsubscript{3} and i\textsubscript{4} the current through M3 and M4;
	\item i\textsubscript{6} and i\textsubscript{7} the current through M6 and M7;
	\item i\textsubscript{8} and i\textsubscript{9} the current through M8 and M9;
	\item i\textsubscript{o1} the current through R\textsubscript{L1};
	\item i\textsubscript{o2} the current through R\textsubscript{L2};
	\item I\textsubscript{D} the bias current;
	\item i\textsubscript{od} the output differential bias current;
	\item $S(t)=\frac{1}{2}-\frac{2}{\pi} \sum_{n=1,3,5 \dots}^{} \frac{1}{n}\sin(n\omega_{LO} t)$ is the switching function of the mixing cell (then we consider the LO stage as it was made up of ideal switches).
\end{itemize}
The differential input RF signal is given by:
\begin{equation}
\label{eq_inRFSignal}
v_{RF}(t)=\frac{V_{RF}}{2}\cos (\omega_{RF}t) 
\end{equation}
Currents flowing into the two RF stages are given by:
\begin{align}
i_3(t) &= I_D + i_d(t)\\
i_4(t) &= I_D - i_d(t)
\end{align}
where, from equation \ref{eq_degenGain}, the $i_d$ current is a small signal variation given by:
\begin{align}
i_d(t)&=I_{RF}\cos (\omega_{RF}t) \notag \\
&= G_{m3,4}\frac{V_{RF}}{2}\cos (\omega_{RF}t) 
\end{align}
It is now possible to define the current flowing into the switching stage recalling that they are discontinuous functions modulated by the switching function S(t):
\begin{align}
 i_6(t) &= [I_D + i_d(t)]S(t) \\
 i_7(t) &= [I_D + i_d(t)]S(t-T_{LO}/2)
\end{align}
and 
\begin{align}
i_8(t) &= [I_D - i_d(t)]S(t-T_{LO}/2) \\
i_9(t) &= [I_D - i_d(t)]S(t)
\end{align}
because M6 and M9 share the same LO signal, the same holds for M7 and M8. 
Hence the current through R\textsubscript{L1} and R\textsubscript{L2} is given by:
\begin{align}
i_{o1}(t) &= I_6(t)+I_8(t) \notag \\
&=[I_D + i_d(t)]S(t)+[I_D - i_d(t)]S(t-T_{LO}/2) 
\end{align}
and 
\begin{align}
i_{o2}(t) &= I_7(t)+I_9(t) \notag \\
&=[I_D + i_d(t)]S(t-T_{LO}/2)+[I_D - i_d(t)]S(t) 
\end{align}
Hence, the output differential current is:
\begin{align}
\label{eq_Iod1}
i_{od}(t) &= I_{o1}(t)-I_{o2}(t) \notag \\
&=2i_d(t) S(t)-2i_d(t)S(t-T_{LO}/2) 
\end{align}
Noticing that:
\begin{align}
\label{eq_SwitchingEquality}
S(t-T_{LO}/2) &=\frac{1}{2}-\frac{2}{\pi} \sum_{n=1,3,5 \dots}^{} \frac{1}{n}\sin(n\omega_{LO} t-T_{LO}/2) \notag \\
&= \frac{1}{2}+\frac{2}{\pi} \sum_{n=1,3,5 \dots}^{} \frac{1}{n}\sin(n\omega_{LO} t) \notag \\
& =1- \frac{1}{2}+\frac{2}{\pi} \sum_{n=1,3,5 \dots}^{} \frac{1}{n}\sin(n\omega_{LO} t) \notag \\
&= 1 - S(t)
\end{align}
and substituting equation \ref{eq_SwitchingEquality} in \ref{eq_Iod1}:\footnote{$cos(x)sin(y)=\frac{1}{2}sin(x+y)+\frac{1}{2}sin(x-y)$}
\begin{align}
\label{eq_Iod2}
i_{od}(t) &= -\frac{8i_d(t)}{\pi}\sum_{n=1,3,5 \dots}^{} \frac{1}{n}\sin(n\omega_{LO} t) \notag \\
&= -\frac{8I_{RF}}{\pi}\cos(\omega_{RF}t)\sum_{n=1,3,5 \dots}^{} \frac{1}{n}\sin(n\omega_{LO} t) \notag \\
&= -\frac{8I_{RF}}{\pi}\sum_{n=1,3,5 \dots}^{} \frac{1}{n}\cos(\omega_{RF}t)\sin(n\omega_{LO} t) \notag \\
&= -\frac{8I_{RF}}{\pi}\sum_{n=1,3,5 \dots}^{} \frac{1}{n2} \sin[(\omega_{RF}+n\omega_{LO})t]+\sin[(\omega_{RF}-n\omega_{LO}) t]
\end{align}
Considering only the fundamental tone of the LO signal (other frequency terms are out-of-band with respect to IF) and multiplying by $R_L=R_{L1}=R_{L2}$ one finds that the differential output voltage is given by:
\begin{align}
v_{od}(t)& = -\frac{4I_{RF}R_L}{\pi}\{\sin[(\omega_{RF}+\omega_{LO}) t]+\sin[(\omega_{RF}-\omega_{LO}) t]\} \notag \\
& = -\frac{4I_{RF}R_L}{\pi}[\sin(\omega_{HF} t)+\sin(\omega_{IF} t)]
\end{align}
then, referring to the IF component:
\begin{align}
v_{od,IF}(t)& = -\frac{4I_{RF}R_L}{\pi}\sin(\omega_{IF} t) \notag \\
& = -\frac{2G_{m3,4}R_LV_{RF}}{\pi}\sin(\omega_{IF} t) 
\end{align}
that eventually yields the Gilbert cell mixer conversion gain:
\begin{equation}
A_{vC} = -\frac{2}{\pi}\frac{R_L}{\frac{1}{g_{m3,4}}+R_S}
\end{equation}
It is now possible to highlight that:
\begin{itemize}
	\item in first approximation the conversion gain does not depends on the amplitude of the LO signal;
	\item both LO and RF components are rejected at the output;
	\item it is possible to keep only the IF component by filter out the HF signal;
	\item the degeneration resistance R\textsubscript{S} improve the stage's linearity. In fact, if properly chosen, one has: $A_{vC}\simeq\frac{2}{\pi}\frac{R_L}{R_S}$;
	\item besides to linearity the conversion gain is less sensitive with respect to the RF stage bias point;
	\item no information about frequency behaviour appear with this kind of analysis.
\end{itemize}  



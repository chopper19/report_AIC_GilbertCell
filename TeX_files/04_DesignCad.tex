\section{Design by simulation}

\subsection{Gilbert Cell design}
After simulation of the design made with Level 1 model, simulation results proven to be not enough accurate for the circuit to behave correctly.
For this reason a new design was carried on taking advantage of the simulated trans-characteristic of each stage.\\
The starting specification was to maximize \(g_{m3}\) of the input RF stage, with some reasonable supposition on transistor dimensions and power consumption:
(CONSIDERAZIONI SU Wopt ? -> troppo grande, grnadezza massima ragionevole 500um)

\begin{align}
	&L=3\cdot L_{min} = 1.8\mu m \nonumber\\
	&50 \mu m \le W_3 \le 500 \mu m \nonumber \\
	&I_0 \approx 5mA \nonumber \\
\end{align}
Further suppositions on bias voltages were made. For example, a voltage equal to \(\frac{3}{5}\cdot V_{dd}\) was supposed to fall between the RF stage drain and ground, equally divided between M1 and M3 (neglecting the very small voltage fall on \(R_S\)). 

\begin{align}
	V_{SB3} &= V_{DS1} = V_{DS3} = 1.5 V \nonumber \\
\end{align}
This way we could take into account the body effect on threshold voltage of transistor M3.
Then we plotted the current and transconductance of the transistor as a function of \(V_{GS}\):

\begin{figure}[H]
	\centering
	\includegraphics[scale=0.6,trim=10cm 10cm 10cm 10cm]{"W2_id"}
	\caption{Drain current versus gate-source voltage of the RF stage, with W varying from 50\(\mu\)m to 500\(\mu\)m}
	\label{W2_id}
\end{figure}
\begin{figure}[H]
	\centering
	\includegraphics[scale=0.6,trim=13cm 10cm 10cm 13cm]{"W2_gm"}
	\caption{Transconductance versus gate-source voltage of the RF stage, with W varying from 50\(\mu\)m to 500\(\mu\)m}
	\label{W2_gm}
\end{figure}

As can be seen from figures \ref{W2_id}, \ref{W2_gm}, the largest width  \(W_3=500\mu m\) (the green curve) was chosen, along with a \(V_{GS3}=1.5V\). This led to the choice of transconductance and drain current:

\begin{align}
	g_{m3}&=11.9mS \nonumber \\
	\frac{I_0}{2}&=2.9mA \nonumber
\end{align}

We moved on to the design of the bias current sink M1. Given the data:
\begin{align}
	V_{R_S} = 2.9mA*10\ohm=29mV
	V_{DS1}=1.5V-V_{R_S}=1.471V
	V_{GS1}=1.471V
\end{align}
We found a width of \(W_1=373\mu m\) in order to have \(I_0=2\cdot2.9mA=5.8mA\):
\begin{figure}[H]
	\centering
	\includegraphics[scale=0.6,trim=12cm 10cm 9cm 10cm]{"W1_id"}
	\caption{Drain current versus gate-source voltage of the bias transistor M1, with W=\(373\mu m\)}
	\label{W1_id}
\end{figure}

Finally the LO stage was designed, along with the load resistance \(R_L\) starting from the wanted conversion gain \(A_v=4\). Given the analytic expression of voltage conversion gain (REF: CALCOLO DEL GUADAGNO DI CONVERSIONE CON LO COMMUTATO):
\begin{align}
	A_v \approx \frac{2}{\pi}\left( \frac{R_L}{R_S + \frac{1}{g_{m3}}}\right)=4 \nonumber
\end{align}
The value of \(R_L\) is determined
\begin{align}
	R_L=A_v \cdot \left( \frac{\pi}{2} \cdot\frac{1}{g_{m3}} + R_S \right)=577 \ohm
\end{align}
The drain source voltage of LO, \(V_{DS6}\) stage can be thus evaluated:
\begin{align}
	&V_{SB6}=V_{DS1}+V_{R_S}+V_{DS3}= 1.47V+0.029V+1.5V=3V \nonumber \\
	&V_{R_L}=2.9mA\cdot 577\ohm=1.673V \nonumber \\
	&V_{DS6}=V_{dd}-V_{R_L}-V_{SB6}=327mV \nonumber
\end{align}

LO gate bias voltages must be slightly above threshold, in order to let the transistors turn on and off with small LO signal variations, deviating rapidly current coming from the RF stage. To accomplish this, threshold voltage of the transistor was extracted from simulation, using the \(g_{m}\) graph as a function of \(V_{GS}\):
\begin{figure}[H]
	\centering
	\includegraphics[scale=0.6,trim=14cm 10cm 10cm 10cm]{"M6_Vth"}
	\caption{Extrapolation of M6 threshold voltage from transconductance versus \(V_{GS}\) curve. The threshold happens to be at \(1.27V\).}
	\label{M6_Vth}
\end{figure}
Having \(V_{th6}=1.27V\), a very small overdrive of \( \Delta V_6=60mV \) was chosen, for the reason reported before, and to assure the stage is not working in triode region. This led to:
\begin{align}
	V_{GS6}=V_{th6}+\Delta V_6=1.33V \nonumber
\end{align}
Once fixed node voltages and the current flowing in the transistor (\(2.9mA\), due to the fact that only two transistors of LO stage conduct simultaneously), transistor width \(W_6\) was swept to fulfill precisely this biasing. The followed procedure is the same displayed in figure \ref{W2_id} and the resulting dimensions found are:
\begin{align}
	&W_6=170.3\mu m \nonumber\\
	&L = L_{min} = 0.6\mu m \nonumber
\end{align}                                                                      
Only for this stage a minimum channel length was taken, in order to keep small these transistors.
                                                                                 
\subsection{Biasing network design}                                              
From the previous section, bias network specification required by current sink, RF and LO stage are suddenly imposed:
\begin{align}                                                                    
	V_{G1}&=1.471 V \nonumber \\                                                    
	V_{G3}&=3 V \nonumber \\                                                        
	V_{G6}&=4.33 V \nonumber                                                        
\end{align}                                                                      
The bias network circuit employed is visible in \\FIGURE HERE:\\                 
Mirroring ratio for the current sink transistor M1 was chosen to be 1. With a transistor length \(L_2 = 1.8\mu m\), and being \(V_{GS2}=V_{DS2}=1.471V\) imposed, \(W_2\) has been trimmed in the simulator to sink a current equal to \(I_0=5.8mA\), resulting in:
\begin{align}                                                                    
	W_2=373\mu m\nonumber                                                           
\end{align}                                                                      
In the same way transistor M5 width was set, in order to have a gate voltage \(V_{G5}=3V\) when stacked above M2:
\begin{align}
	W_5&=130.45\mu m \nonumber\\
	L_5&=0.6\mu m \nonumber
\end{align}
Minimum length was taken in this case for reduced transistor size.
The sum of \(R_2\) and \(R_4\) must be such that with a current of \(5.8mA\) the voltage drop across them is such that \(V_{G5}=3V\):
\begin{align}
	R_2+R_4=\frac{V_{dd}-V_{G5}}{I_0} = 344\ohm \nonumber
\end{align}
The partition between them must give a bias voltage to LO stage of \(4.33V\). This brings to
\begin{align}
	R_2=229\ohm\nonumber\\
	R_4=115\ohm\nonumber
\end{align}

Resistors \(R_1\) and \(R_3\) were chosen to be large enough to isolate biasing network from RF and LO signals respectively, and form a low pass filter along with capacitors \(C_1\) and \(C_2\). Resistors were thus chosen of the order of some k\ohm.
\begin{align}
	R_1=R_3=30k\ohm \nonumber
\end{align}
From here minimum capacitance to filter LO signal can be evaluated in order to have a pole at least one decade before working frequency. The output resistance of M5 has been considered negligible, leading to an equivalent resistance for the low pass filter of:
\begin{align}
	R_{eq}=R_1//R_3//R_2//R4=76.4 \ohm \nonumber \\
\end{align}
from which the resulting capacitance:
\begin{align}
	C_1\ge 10\cdot \frac{1}{2\pi \cdot R_{eq} \cdot f_{lo }} = 20.8pF
\end{align}
\(C_1 = C_2\) was chosen for simplicity, and a sweep over them was carried on to maximize the gain, leading to the final chosen value of:
\begin{align}
	C_1=C_2=25pF
\end{align}